%reaching back to introduction
%suggest improvements
%(highlight importance of results)

\section{Conclusion}
\label{sec:Conclusion}

The manufactoring of microelectronics is a important pillar of todays digital society, and the stable characteristics of nanofabricated elements make sure reproducabilty and functionality.
Here, we have sucessfully produced a photodetector element based on a photoconductor architecture in a bottom up nanofabrication procedure.
Electric contacts could be fabricated using photolihography.
A perovskite has been used as an active layer while it has been hardend in to different ways. 
Once, it was deposited and baked out, the other version was using an antisolvant to force the crystallization of the active layer.
By optical and electric characterization of both samples the full functionality of the samples as an on/off but also quantitative photodetector could be confirmed.
By comparing the two samples and thus production strategies, differences in the policrystalin element size has been detected.
This, plausibly has also an effect on the electronic properties of the detector as the sample produced with an antisolvant, having small crystalls only, has a higher slope of conductivity depending on the gate voltage.
This demonsrates, that in nanofabrication also small differences of production parameters can have reasonable benefits but also unwished effects.



