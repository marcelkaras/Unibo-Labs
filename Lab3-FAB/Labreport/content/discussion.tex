%interpretation
%comparing and contrasting to litearture
%explain errant data, find possible error sources
%What do the results mean?
%(May it be added in the results"?)
\section{Discussion}
\label{sec:Discussion}

% \begin{align*}
%     \delta E &= \frac{E_\text{LED} - E_\text{Laser}}{E_\text{LED}} \\
%     \delta E_\text{A} &= \SI{2.4}{\percent} \\
% \end{align*}

As already intoduced in the results \ref{sec:Results}, it becomes clear that both samples A and B are sufficient to be used not only as an on/off photodetector, which is undergird by the time resolved gate current measurement \ref{fig:dynamic},
but also in a quantitive way, as the gate resistance is almost linear (compare with figure \ref{fig:hyst-A},\ref{fig:hyst-B}).
Although both samples seem to be sufficent enough to be used as a photodetector there are several differences due to the production methodology.
The observation of smaller crystals and a more intense colour of sample B (figure \ref{fig:optic-B} and \ref{fig:hyst-B}) can be bring together as the smaller cristals offer a bigger surface and thus to more diffraction.
This brings together the small appearance of the crystals and the more intense color in sample B compared to sample A.
We know, that the antisolvant used for producing sample B forces an almost instantanous crystallization.
Thus, the crystals do not have the time to grow as basically every molecule is a grain of crystal growing.

Surely, from the different policrystalin dimmensions one can espect different electronic properties, as domain walls have the effect of defects.
Indeed, we have obeserved that the slope of samples B conductivity is higher in the luminated case.
This may lead to a more sensitive, quantitative device.