%reaching back to introduction
%suggest improvements
%(highlight importance of results)

\section{Conclusion}
\label{sec:Conclusion}

The appealingly easy and macroscopic measurement of resistance, as this experiment shows can provide microscopic information about a given sample.
Not only could be understood from microscopic models of intrinic and extrinsic charge carrier density and their dependency on temperature, why metals and semiconductors have very different characterisics,
but also material parameters as the semiconductors bandgap, here for germanium could be estimated in a precise way.
Also the proportonality parameter $\alpha$ for germanium could be found.

Further, in a second part of the experiment three of four unknown samples could be identified by estimating there resistivity, just measuring the wires lenght and thickness such as the resistance.
For sample A the fitting was not sufficient which is understandable, as the thin wire was bad to handle.
A shortcut with the environment or a bad contact to the wire may have occured  which manipulated the results.
But as the estimate of sample D was copper, where also the specific color of the wire lets to guess this material, one can see this as a proof, that at least for sample D, the estimate done by using the resistivity is correct.

Not only the above given results of the done experiments are important, but also that three different ways of resistance measurement have been used and compared to each other in a technical and precision way.


