%interpretation
%comparing and contrasting to litearture
%explain errant data, find possible error sources
%What do the results mean?
%(May it be added in the results"?)
\section{Discussion}
\label{sec:Discussion}

% \begin{align*}
%     \delta E &= \frac{E_\text{LED} - E_\text{Laser}}{E_\text{LED}} \\
%     \delta E_\text{A} &= \SI{2.4}{\percent} \\
% \end{align*}

First of all, resistance characteristics of matalls will be discussed.
Here, a linear dependency of the resitance $R$ of the temeprature $T$ could be found (\ref{equ:metalic-fit}) with simmilar parameters $\beta$ for both samples.
In order to derive the resistivity $\rho(T)$ of the material, just the factor $L/S$ has to be considered,
which is 
\begin{equation}
    \rho(T) = R(T) \times S/L.
\end{equation}
As the length of the sample is not known, a further analysis can not be done in the absolute regime.
But we know, that the reduced resistance is equal to the reduced resistivty.
%Maybe I should add some calculations of the reduced resistance = resistivity

Comparing the slope, which is the temperature coefficient $\beta$ \ref{equ:results-beta}, one can see it is bigger for copper compared with nickel, which fits to the data of Wiser \cite{resistivity}.
For low temperatures, also the semiconductor sample of germanium can be fitted lineary, resulting in a temperature coefficient in the same scale \ref{equ:results-beta}.
This behaviour can be explained by the Drude model with a constant density of nearly free charge carriers, which are in the case of the semiconductor germanium the electrons from the extrinsic region.

In doped semiconductors as in the N-doped germanium sample the carriers come from thermally generated, intrinsic electron-hole pairs and extrinsic carriers due to the dopant.
As seen above, the extrinsic carriers density is temperature independant.
Thus, the resistance depends on the temperature dependent carrier`s mobility $\mu$.
Taking into account these both effects, the resistance $R$ is proportional to $T^\alpha$, with a typical value of $\alpha = 1.5$ for germanium.
By a linear regression in a double logarithmic plot this value could be found very precisely \ref{equ:alpha} with a error of
\begin{equation}
    \Delta \alpha = \frac{\alpha_\text{lit}-\alpha_\text{exp}}{\alpha_\text{lit}} = \SI{0.1 \pm 0.3}{\percent}.
\end{equation}

Further, the intrinsic carriers depend exponentially on the temperature, what could been shown for high temperature, where the amount of intrinsic carriers is much higher then the one of extrinsic carriers.
The resulting calculated bandgap of germanium \ref{eq:bandgap-exp} can be compared with the literature banggap of undoped germanium, which is $\SI{0.67}{\eV}$ \cite{bandgap-Ge}.
The resulting discrepancy is
\begin{equation}
    \Delta E_g = \frac{E_\text{lit}-E_\text{exp}}{E_\text{lit}} = \SI{24.5 \pm 0.8}{\percent}.
\end{equation}

Now, the low and also the high temperature regime have been analyzed sufficcently. 
For the gap between, the transition phase, there is an overlap of both models thus an analysis is not easy to do.
Generally, one can say that the transition temperature depends on the dopant concentration, where it shifts to higher temperatures for higher concentration.

\subsection{Resistivity of wires}
\label{sec:wires-disc}

The resistivity parameter could be fitted by using pairs of wire length and resistance measurement.
Firstly, one can assume that the 4-wire measurent is better as the measure error of wire resistance could be overcome.
Therefore, the discrepancy between both results can be calculated.
The results are displayed in table \ref{tab:wires} and the relative error is generally very big (procentual deviation $>\SI{31}{\percent}$)which shows the importance of using the more precise, 4-wire measurement technique.
By camparing the resistivity with values from the literature, an resonable estimate for the samples material can be done.
For the chosen materials 
\begin{align*}
    A -& \text{none} &\\
    B -& \text{magnese} &-\rho=\SI{144e-8}{\ohm\meter}  \\
    C -& \text{lead} &- \rho=\SI{22e-8}{\ohm\meter} \\
    D -& \text{copper} &- \rho=\SI{1.68e-8}{\ohm\meter}  \\
\end{align*}\cite{lead}\cite{copper}
another procentual calculation gives the error of
\begin{align*}
    \Delta \rho_\text{B-Mg} &= \SI{17 \pm 4}{\percent} \\
    \Delta \rho_\text{C-Pb} &= \SI{10 \pm 22}{\percent} \\
    \Delta \rho_\text{D-Co} &= \SI{174 \pm 12}{\percent} .\\
\end{align*}
For the sample A the fitting was unsufficient as on the one side there is no metal with a comparable resisitvity and on the other side one can see a wide spread of the measurement point, and so a high standard derivation of the estimated resisitvity.
