%interpretation
%comparing and contrasting to litearture
%explain errant data, find possible error sources
%What do the results mean?
%(May it be added in the results"?)
\section{Discussion}
\label{sec:Discussion}

% \begin{align*}
%     \delta E &= \frac{E_\text{LED} - E_\text{Laser}}{E_\text{LED}} \\
%     \delta E_\text{A} &= \SI{2.4}{\percent} \\
% \end{align*}

First of all, resistance characteristics of matalls will be discussed.
Here, a linear dependency of the resitance $R$ of the temeprature $T$ could be found (\ref{equ:metalic-fit}) with simmilar parameters $\beta$ for both samples.
In order to derive the resistivity $\rho(T)$ of the material, just the factor $L/S$ has to be considered,
which is 
\begin{equation}
    \rho(T) = R(T) \times S/L.
\end{equation}
As the length of the sample is not known, a further analysis can not be done in the absolute regime.
But we know, that the reduced resistance is equal to the reduced resistivty.
%Maybe I should add some calculations of the reduced resistance = resistivity

Comparing the slope, which is the temperature coefficient $\beta$ \ref{equ:results-beta}, one can see it is bigger for copper compared with nickel, which fits to the data of Wiser \cite{resistivity}.
For low temperatures, also the semiconductor sample of germanium can be fitted lineary, resulting in a temperature coefficient in the same scale \ref{equ:results-beta}.

