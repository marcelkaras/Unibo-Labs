\section{\label{sec:Formatting}Formatting}

\ptitle{Formatting checklist} Whether you are using a compiler on your computer or online, please use the latest version of REVTeX, and check your formatting carefully.
\begin{itemize}[label=$\Box$]
\item Check math \& symbolic formatting, as in Table \ref{tab:mathformat}.
\begin{table}[h!]
  \begin{center}
    \caption{Formatting mathematical symbols.}
    \label{tab:mathformat}
    \begin{tabular}{c|c} % <-- Alignments: l for left, c for center, and r for right, with vertical lines in between
      \hline
      \textbf{Incorrect} & \textbf{Correct} \\
      \hline \hline
      $cos \theta$ & $\cos \theta$ \\
      $T_{sample}$ & $T_\mathrm{sample}$ \\
      V$_{rms}$, V (rms) & $V_\mathrm{rms}$ \\
      $E_\mathrm{x}$, x direction & $E_x$, $x$ direction \\
      $\vec{B_\mathrm{app}}$ & $\vec{B}_\mathrm{app}$ \\
      $Sb_2Te_3$, Sb2Te3 & Sb$_2$Te$_3$ \\
      Sb$_{2-\mathrm{x}}$V$_\mathrm{x}$Te$_3$ & Sb$_{2-x}$V$_x$Te$_3$ \\
      dI/dV & $dI/dV$ \\
      $B = 5 T$, B=5T & $B=5$ T \\
      x direction, X direction & $x$ direction \\
      1st, $1^{st}$, 2nd, $2^{nd}$ & 1$^\mathrm{st}$, 2$^{\mathrm{nd}}$ \\
      \hline
    \end{tabular}
  \end{center}
\end{table}
\item Use {\tt \textbackslash label\{tab:name\}} and {\tt \textbackslash ref\{tab:name\}} to refer to tables.
\item Check hyphenation. Sometimes \LaTeX\ likes to divide a single letter off the beginning or end of a word, for line wrapping. The default settings for \LaTeX's hyphenation of English-language words are {\tt \textbackslash righthyphenmin=3} and {\tt \textbackslash lefthyphenmin=2}, but apparently they can be mysteriously reset to allow single dangling letters.
\item Check spacing. When a period falls in the middle of a sentence, use a {\tt \textbackslash} (backslash) to prevent \LaTeX\ from thinking it's the end of the sentence and thus adding extra space, as shown in Table \ref{tab:spacing}. If you want to prevent a linebreak, you can use {\tt \textasciitilde} instead of {\tt \textbackslash}.
\begin{table}[h!]
  \begin{center}
    \caption{Spacing.}
    \label{tab:spacing}
    \begin{tabular}{l|c|c} % <-- Alignments: l for left, c for center, and r for right, with vertical lines in between
      \hline
       & \LaTeX & Output \\
      \hline \hline
      Incorrect & {\tt e.g.\ incorrect} & e.g. \ incorrect \\
      Incorrect & {\tt Fig.\ 2} & Fig. \ 2 \\
      Correct & {\tt e.g.\textbackslash\ correct} & e.g.\ correct \\
      Correct & {\tt Fig.\textbackslash\ 2} & Fig.\ 2 \\
      Correct & {\tt Fig.\textasciitilde 2} & Fig.~2 \\
      \hline
    \end{tabular}
  \end{center}
\end{table}

\item Number all equations. But do not separately number each line of a single multi-line equation.
\item Use {\tt \textbackslash label\{eqn:name\}} and {\tt \textbackslash ref\{eqn:name\}} to refer to equations.
\item If the equation is mid-paragraph, use {\tt \textbackslash noindent} at the beginning of the first line following the equation.
\end{itemize}
\vspace{2mm}

\ptitle{Equations} Here is an improperly-labeled equation in the middle of a paragraph.
% Simple equation with no number; this is not a good idea!
\[
1+1=2
\]
\noindent Use {\tt \textbackslash noindent} to prevent indentation mid-paragraph. Here is a more interesting example of a properly labeled equation: the Pythagorean theorem relates the 3 sides of a right triangle according to Eqn.\ \ref{eqn:Pythagoras},
% Here is a proper equation with a number and a label, so that it can be referenced later.
\begin{equation}
a^2+b^2=c^2 \,.
\label{eqn:Pythagoras}
\end{equation}
\noindent Eqn.\ \ref{eqn:diagonal} shows one more example of a multi-line equation extending the Pythagorean theorem to find the diagonal $d$ of a rectangular prism of sides $a=3$, $b=4$, and $c=12$.
\begin{eqnarray}
\label{eqn:diagonal}
\nonumber d & = & \sqrt{a^2 + b^2 + c^2} \\
& = & \sqrt{3^2+4^2+12^2} = 13
\end{eqnarray}