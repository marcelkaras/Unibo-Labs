\maketitle
\section{\label{sec:Start}Getting Started}

\ptitle{Start writing while you experiment} You should start writing your paper \textit{while} you are working on your experiment. Prof.\ George Whitesides says: ``A paper is not just an archival device for storing a completed research program; it is also a structure for planning your research in progress. If you clearly understand the purpose and form of a paper, it can be immensely useful to you in organizing and conducting your research. A good outline for the paper is also a good plan for the research program. You should write and rewrite these plans/outlines throughout the course of the research. At the beginning, you will have mostly plan; at the end, mostly outline. The continuous effort to understand, analyze, summarize, and reformulate hypotheses on paper will be immensely more efficient for you than a process in which you collect data and start to organize them only when their collection is `complete' .'' Here are some concrete steps to get started.

\begin{enumerate}
\item Read George Whitesides' ``How to Write a Paper'' \cite{WhitesidesAdvMat2004}.

\item Read through \emph{at least} one full paper in your target journal, to get a sense of the content and writing style.

\item To clarify in your own head the purpose of your paper, start by writing your abstract.

\item Before you tackle the body of the paper, sketch block outlines of the figures. Decide what images and plots you will put in the paper, and how the panels will be arranged.

\item Outline at the paragraph level before you write. Look at how many paragraphs there will be in the style of paper you are trying to write. For example, for a standard 4-page scientific letter, aim for 13 paragraphs (generally, you can estimate about 200 words per paragraph). Figure out how to tell your entire story (not numbers, just story!) in 13 stand-alone sentences.

\item Make each of those sentences into the first sentence of a paragraph, and fill into each paragraph only details that are relevant to that first sentence. If you find yourself writing details about the figures, cut and paste them into the captions.

\item If you think of references as you go, you can include the minimal identifying information in parentheses to trigger your memory later, e.g.\ ``(WhitesidesAdvMat)'', so all of the information is compact.

\item Rewrite your abstract, taking into account what you have learned from the process of writing the paper. As you fine-tune your abstract, you may find it helpful to refer to Nature's instructions for writing an abstract \cite{NatureAbstract} and for clear communication more generally \cite{NatNeuro2000}.
\end{enumerate}
\vspace{2mm}

\ptitle{Formatting matters} As you contemplate the paper you have just written, put yourself in the shoes of the reviewers (including your collaborators). You already work many, many hours/week, and you don't really want to spend more time reading this paper. So you're going to be very happy if the figures are pretty, the text flows logically, the references are hyperlinked for easy access, and you can understand the paper quickly. But you're going to be very grumpy if you can't get the main points of the paper from scanning through the figures \& captions. You're going to be even grumpier if you invest time in reading the paper but you still can't get it. Your evaluation of this paper is likely to be swayed by your ease of understanding, regardless of the scientific merits of the work. (See Ref.\ \onlinecite{Kahneman2011} for more information on how formatting, even as simple as font choice, will influence the reader's ``cognitive ease'' and ultimately their judgment of the report.) Down the road, consider a reader who might cite the paper and launch you to fame and glory: the potential citer's decision will be influenced by their ability to easily understand your paper.

\ptitle{Fractal} \textbf{Your paper should be fractal.} Somebody with 30 seconds to look at it should be able to get the main idea just from reading the abstract. Somebody with 5 minutes should be able to look at the figures and captions and get more out of it. Somebody with 10 minutes should be able to get the story from the first sentence of each paragraph.

