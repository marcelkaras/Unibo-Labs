%interpretation
%comparing and contrasting to litearture
%explain errant data, find possible error sources
%What do the results mean?
%(May it be added in the results"?)
\section{Discussion}
\label{sec:Discussion}

% \begin{align*}
%     \delta E &= \frac{E_\text{LED} - E_\text{Laser}}{E_\text{LED}} \\
%     \delta E_\text{A} &= \SI{2.4}{\percent} \\
% \end{align*}

Although the results from the fluorescence and diffraction measurement seem to be missleading, a reasonable argumentation of bringing them together can be found.
First of all, the assumed $\ce{Ni}$ line from the fluorescence measuremnt is very close to the $\ce{Cu}$ line, and may be derivated from it, either as a peak in second order, or a missing signal in the middle between the both peaks.
Generally one can say, that chemical bonds can move peaks in the fluorescence spectrum as the bonding length changes the bandstructure of the material.
Following this argumentation, small deviations in the energetic position of the characteristic peaks are not relevant in order to choose candidates for the elements.

Further, the diffraction spectra comparrission shows once, the $\alpha$ peak of cupper is fitting to the most intense response energy in the measured spectrum. 
This adds confidentiallity to the assumption of a $\ce{Cu}$ compound in the sample B.
As mentioned above, one have to assume that due to the atmospherical conditions the copper may oxidate. 
Indeed, several peaks of the measurement can be matched to this chemical state of cupper and oxigen. 
Taking into account all these considerations, it is reasenable to make the assumption that sample B is a mixture of cupper $\ce{Cu}$ and cupper oxide $\ce{CuO}$.