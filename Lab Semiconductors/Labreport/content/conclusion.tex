%reaching back to introduction
%suggest improvements
%(highlight importance of results)

\section{Conclusion}
\label{sec:Conclusion}

In the application of organic semiconductors in the field of spintronics two relatively new and interdisciplinar branches of physics come together.
We have seen that spintronics, where spin states of charge carriers can be used additionally (or seperately) to the charge in order to store and process information.
This is very useful considering the demand of faster, smaller and less energy-consuming electronics.
We have seen that one of the biggest problems in designing a fully functional, easy producible spintronics device, is the small propagation length of spin-states in conventional semiconductors.
Here, organic semiconductors, as they have a low spin-orbit coupling due to their light elements, easy adoptable molecular structure and packing are promoting candidates for waveguides.

But not only that, as especially the optoelectric properties of OSCs can be combined with the spintronic function, so completely new functionalities can arise.
Nevertheless, manufacturing difficulties, such as the surface between metallic and organic elements have been discussed, as well as strategies to solve them.

In the end, the exemplary spin-OLED device has been presented, which allows to improve the luminescence compared to standard OLEDs, already \cite{spin-OLED} \cite{appl-organic}.

Only the future can show us, if the branch of organic spintronics will be a successful one, as the parameters of physical functionality always have to be balanced with the economical expenses.
Also other groups of materials, such as perovskites are under research currently and may outrun the opportunities of OSCs, as they are cheap and easier to fabricate in thin layers \cite{perovskite}.




