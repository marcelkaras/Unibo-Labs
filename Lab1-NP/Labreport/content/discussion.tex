%interpretation
%comparing and contrasting to litearture
%explain errant data, find possible error sources
%What do the results mean?
%(May it be added in the results"?)
\section{Discussion and Conclusion}
\label{sec:Discussion}

% -thermal broadaning of lambda peak
% -colloidal QD dont have all the same size -> different lowest CB energy -> different bandgap

Using a rather simple measurement technique measuring a photoluminescence spectrum, this experiment shows clearly it is possible to analyze the bandgap of semiconductors.
Just adding the parameter of quantum confinement, which shall be the same for all three samples, the chemical characterization became possible.
Here, the proportion of Sulphur and Selenium inside a compound could be related to the band gaps. 
Also the bandgap of the perskite sample could be measured.

Although the background correction and signal integration reduced the noice of the signal sufficcient enough for finding accuratly the peak wavelength, both for the LED and Laser probed measurements,
especially the error of a calculated concentration greater 1 occurs, which definetlly can not correspond with the reality.
As the calculation bases on the assumption of a quantum confinement of $\SI{6}{\nano\meter}$, which has not been derived from an experiment, one could expect a wrong estimate.
This wrong estimate can easily shift the calculated concentration in its value out of logical range.
Surely, this missbehavior for the sample A calculation, derives a mistrust in the calculations for sample B and C, but can not be proven.
Instead, for improving the quantitative quality of the calculations, a proof for the quantum confinement could be done.

Comparing the LED and laser excited spectra for all the samples, only a small procentual variance can be found, which shows, that for the samples A,B and C but not for the perovskite the spectral confinement, as well as the power of the laser, a LED is sufficient enough.
The deviation is 
\begin{align*}
    \delta E &= \frac{E_\text{LED} - E_\text{Laser}}{E_\text{LED}} \\
    \delta E_\text{A} &= \SI{2.4}{\percent} \\
    \delta E_\text{B} &= \SI{0.2}{\percent} \\
    \delta E_\text{C} &= \SI{0.2}{\percent}. \\
\end{align*}

Further improvements could be optained by low temperature measuerments, which allow to surpress thermal broadaning effects on the spectra, due to broadaning of the bands.
