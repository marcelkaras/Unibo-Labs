% background
% brief review of previous research (cite)
% reson why the research was undertaken
% Hypothesis
% explenation of techniques and why they ve been chosen
% objectives = what you hope to achieve
% brief reference to the main outcome
\section{Introduction}
\label{sec:Introduction}

Luminescance is a phenomenon where a sample of any kind get energetically excited and thus emmits light of a material and structual specific spectrum.
But also it is used in nowaday consumer tecnologies like TV screens. 
In this case, photoluminescense is used to charactize the specifications three samples, which are semiconductor compunds of the same elemnts but in different concentrations.
In the following, the samples as well as the process of photoluminescense is recalled.

\subsection{Photoluminescense in Semiconductors}
\label{sec:PLS}

Semiconductors are characterized by a bandgap of $\SI{2}{\eV}$ to $\SI{4}{\eV}$, a almost completely fulled valence band and a almost completaly empty conductance band \cite[391]{festkorperphysik}.
Here, a lightsourse like a LED or Laser is able to excite some electrons from the valence band up into the cunductance band, as long as the photoenergy is bigger then the bandgap of the sample.
The electrons loose energy by exciting lattice vibrations, and by reaching the minimum of the conductance band, they may recombine with a hole in the valence band, emmitting a photon with the energy of the band gap.
Here, the equation
\begin{equation}
    h\nu = E_{\text{g}}
\end{equation}
holds, where $\nu$ is the photons frequency, and $E_\text{g}$ the band gap.
One easily can see, that the expected frequency of the emitted photons is smaller then the freqency of the pump. 
Nethertheless, due to thermal effects a thermal spread along the energy of the bandgap, a widening of the spectrum around a $\lambda_\text{peak}$ is expected.
Using a spectrometer, this wavelengh can be found, and used to estimate the bandgap energy
\begin{equation}
    E_{\text{g}} = \frac{h\c}{\lambda_\text{peak}}.
\end{equation}

\subsection{Samples}
\label{sec:samples}

There are three samples to analyze, where all of them can be described as quantum dots, as the $\ce{\Zn\S}$ has a bigger bandgap than the $\ce{\Cd\S_x\Se_{1-x}}$ molecule, thus it acts like a potential barrier.
This provides a confinement potgential. The spectra of quantum dots are reffered in the chapter \ref{sec:QD}.
Then, the molecules of $\ce{\Cd\S_x\Se_{1-x}/\Zn\S}$ with a unknown concentration $x$ are submered in tulene.

A other sample is methylammonium lead bromide ($\ce{\M\A\Pb\Br_3}$), a hybrid perovskite single crystal 
\subsection{label:QD}

